        
a native MPI library (e.g.\ LAM/MPI or MPICH),
that 
that the MPI runtime system is up and running (e.g.\ using \texttt{lamboot}).
If not please see the installation notes at the pypar web site.

        
%Taken from exercise 5 

\subsection*{Preliminaries}

\subsubsection*{The shell configuration}
Log in using the username you created previously.
You will need to configure a few environment variables to make 
life simpler.
Environment variables control the way you personally interact with the
shell and they are defined in the file \texttt{.bashrc} which lives in
your home directory. Edit \texttt{.bashrc} and add the following lines

\begin{verbatim}
export PATH=.:$HOME/bin:${PATH}
export LAMRSH="ssh -x"
\end{verbatim}

The first line asks the shell to always look for files in the directory
you are working in. The second tells the MPI runtime system to use 
secure shell (ssh) without graphics capabilities (X11). 
Close the shell window and open a new one for this to take effect.  

\subsubsection*{Setting up the secure shell environment}

Next we will make sure that you can access all machines without having to
type in your password every time. The secure shell (ssh) has a capability
to store your password encrypted on your disk and automatically provide
it when needed. To accomplish this you need the following steps:

\begin{enumerate} 
  \item ssh-keygen -t rsa (Generate codes for the encryption)\\
  \emph{When asked for a filename, just go with the suggested default.
  When asked for a passphrase just press \textbf{Enter} twice}
  \item cd \verb+~/.ssh+  
  \item cp id\_rsa.pub authorized\_keys   
\end{enumerate}

Whenever you try to log into a remote host using ssh, your host will 
negotiate with the remote host using these keys and if accepted provide
the password automatically. To test try
\begin{verbatim}
  ssh ninja-n whoami
\end{verbatim}
where 
\[
  n = 1, 2, \ldots 8
\]
This will log you in briefly issue the command \texttt{whoami},
display the result and log you out again.

The first time you do this you will see the message
\begin{verbatim}
The authenticity of host 'ninja-8 (192.168.1.8)' can't be established.
RSA key fingerprint is 06:eb:b3:e0:f1:57:35:0e:58:49:b4:48:46:fb:5c:aa.
Are you sure you want to continue connecting (yes/no)?
\end{verbatim}
which means that this is the first time the encrypted connection 
is to be established. Just type \texttt{yes}.
You might also see a warning message about X11 forwarding which 
you can ignore. From now on you should be able to move freely among 
the hosts in our cluster.
See \texttt{man ssh} for more information.

\subsection*{The MPI runtime system}

We are now ready to start the \emph{MPI runtime system} which comprises a
number of servers (one on each machine) who will allow you to execute
MPI programs in parallel.

The runtime system needs a configuration file listing 
the hosts who are participating in the cluster.
Create a file called \texttt{.lamhosts} in your 
home directory (\verb+~+). 
To begin with list \emph{only} the machine you are working on, e.g.
\begin{verbatim}
ninja-n
\end{verbatim} 
where n is the number you are working on.
Later we will add more machines.
%namely \texttt{ninja-1} to \texttt{ninja-8}.

The MPI runtime is then started by issuing the command
\begin{verbatim}
  lamboot -v ~/.lamhosts
\end{verbatim}
which reads the configuration file, logs in to all listed hosts and 
start the MPI servers.  
You should see a diagnostic message like
\begin{verbatim}
LAM 6.5.8/MPI 2 C++/ROMIO - Indiana University

Executing hboot on n2 (ninja-n - 1 CPU)...
\end{verbatim}
and you are ready to run MPI.

        

