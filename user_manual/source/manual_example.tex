% Complete documentation on the extended LaTeX markup used for Python
% documentation is available in ``Documenting Python'', which is part
% of the standard documentation for Python.  It may be found online
% at:
%
%     http://www.python.org/doc/current/doc/doc.html

\documentclass{manual}


\usepackage{graphicx}
\usepackage{hyperref}
%\usepackage[english]{babel}
\usepackage{datetime}
\usepackage[hang,small,bf]{caption}
\usepackage{amsbsy,enumerate}

\usepackage{amsmath, amssymb, amsthm}

\title{Big Python Manual}

\author{Your Name Here}

% Please at least include a long-lived email address;
% the rest is at your discretion.
\authoraddress{
	Organization name, if applicable \\
	Street address, if you want to use it \\
	Email: \email{your-email@your.domain}
}

\date{April 30, 1999}		% update before release!
				% Use an explicit date so that reformatting
				% doesn't cause a new date to be used.  Setting
				% the date to \today can be used during draft
				% stages to make it easier to handle versions.

\release{x.y}			% release version; this is used to define the
				% \version macro

\makeindex			% tell \index to actually write the .idx file
\makemodindex			% If this contains a lot of module sections.


\begin{document}

\maketitle

% This makes the contents more accessible from the front page of the HTML.
\ifhtml
\chapter*{Front Matter\label{front}}
\fi

%%*************************************************
%     LaTeX document
%
%  This is the ANUGA User Manual
%
%  Version 1.2    July, 2010
%
%
%
%  Copyright Commonwealth of Australia (Geoscience Australia) and the Australian 
%  National University 2004-2010.
%
%  COPYRIGHT PAGE
%
%
%*************************************************

\vspace*{0.5in}

\copyright Commonwealth of Australia (Geoscience Australia) and the Australian 
National University 2004-2010.

Permission to use, copy, modify, and distribute this software for any
purpose without fee is hereby granted under the terms of the GNU
General Public License as published by the Free Software Foundation;
either version 2 of the License, or (at your option) any later
version, provided that this entire notice is included in all copies
of any software which is or includes a copy or modification of this
software and in all copies of the supporting documentation for such
software.
This program is distributed in the hope that it will be useful,

but WITHOUT ANY WARRANTY; without even the implied warranty of
MERCHANTABILITY or FITNESS FOR A PARTICULAR PURPOSE.  See the
GNU General Public License (\url{http://www.gnu.org/copyleft/gpl.html})
for more details.

You should have received a copy of the GNU General Public License
along with this program; if not, write to the Free Software
Foundation, Inc., 59 Temple Place, Suite 330, Boston, MA  02111-1307

This work was produced at Geoscience Australia and the Australian
National University funded by the Commonwealth of Australia. Neither
the Australian Government, the Australian National University,
Geoscience Australia nor any of their employees, makes any warranty,
express or implied, or assumes any liability or responsibility for
the accuracy, completeness, or usefulness of any information,
apparatus, product, or process disclosed, or represents that its use
would not infringe privately-owned rights. Reference herein to any
specific commercial products, process, or service by trade name,
trademark, manufacturer, or otherwise, does not necessarily
constitute or imply its endorsement, recommendation, or favoring by
the Australian Government, Geoscience Australia or the Australian
National University.  The views and opinions of authors expressed
herein do not necessarily state or reflect those of the Australian
Government, Geoscience Australia or the Australian National
University, and shall not be used for advertising or product
endorsement purposes.

This document does not convey a warranty, express or implied,
of merchantability or fitness for a particular purpose.

\begin{center}
  %\vspace{0.5in}
   \anuga

   Manual typeset with \LaTeX
\end{center}

\vspace{0.5in}

\textbf{Credits}:
\begin{itemize}
\item \anuga was developed by Stephen Roberts, Ole Nielsen, Duncan Gray and Jane Sexton. It is currently being developed and
 maintained by Nariman Habili and Stephen Roberts.
\index{ANUGA!credits|textit}  
\end{itemize}

\textbf{License}:
\begin{itemize}
\item \anuga is freely available and distributed under the terms of the GNU General Public Licence.\index{ANUGA!licence|textit}
\end{itemize}

\pagebreak
\textbf{Acknowledgments}:
\begin{itemize}
\item Ole Nielsen, James Hudson, John Jakeman, Rudy van Drie, Ted Rigby, 
      Petar Milevski, Joaquim Luis, Nils Goseberg, William Power,
      Trevor Dhu, Linda Stals, Matt Hardy, Jack Kelly and Christopher
      Zoppou who contributed to this project at various times.
\item A stand alone visualiser (anuga\_viewer) based on Open-scene-graph was developed by Darran Edmundson and James Hudson.
\item The mesh generator engine was written by Jonathan Richard Shewchuk and made freely
      available under the following license.  See source code \code{triangle.c} for more
      details on the origins of this code. The license reads

\begin{verbatim}
/*****************************************************************************/
/*                                                                           */
/*      888888888        ,o,                          / 888                  */
/*         888    88o88o  "    o8888o  88o8888o o88888o 888  o88888o         */
/*         888    888    888       88b 888  888 888 888 888 d888  88b        */
/*         888    888    888  o88^o888 888  888 "88888" 888 8888oo888        */
/*         888    888    888 C888  888 888  888  /      888 q888             */
/*         888    888    888  "88o^888 888  888 Cb      888  "88oooo"        */
/*                                              "8oo8D                       */
/*                                                                           */
/*  A Two-Dimensional Quality Mesh Generator and Delaunay Triangulator.      */
/*  (triangle.c)                                                             */
/*                                                                           */
/*  Version 1.6                                                              */
/*  July 28, 2005                                                            */
/*                                                                           */
/*  Copyright 1993, 1995, 1997, 1998, 2002, 2005                             */
/*  Jonathan Richard Shewchuk                                                */
/*  2360 Woolsey #H                                                          */
/*  Berkeley, California  94705-1927                                         */
/*  jrs@cs.berkeley.edu                                                      */
/*                                                                           */
/*  This program may be freely redistributed under the condition that the    */
/*    copyright notices (including this entire header and the copyright      */
/*    notice printed when the `-h' switch is selected) are not removed, and  */
/*    no compensation is received.  Private, research, and institutional     */
/*    use is free.  You may distribute modified versions of this code UNDER  */
/*    THE CONDITION THAT THIS CODE AND ANY MODIFICATIONS MADE TO IT IN THE   */
/*    SAME FILE REMAIN UNDER COPYRIGHT OF THE ORIGINAL AUTHOR, BOTH SOURCE   */
/*    AND OBJECT CODE ARE MADE FREELY AVAILABLE WITHOUT CHARGE, AND CLEAR    */
/*    NOTICE IS GIVEN OF THE MODIFICATIONS.  Distribution of this code as    */
/*    part of a commercial system is permissible ONLY BY DIRECT ARRANGEMENT  */
/*    WITH THE AUTHOR.  (If you are not directly supplying this code to a    */
/*    customer, and you are instead telling them how they can obtain it for  */
/*    free, then you are not required to make any arrangement with me.)      */
/*****************************************************************************/
\end{verbatim}

\pagebreak
\item Pmw is a toolkit for building high-level compound widgets in
Python using the Tkinter module. Parts of Pmw have been incorpoated
into the graphical mesh generator. The license for Pmw reads

\begin{verbatim}
"""
Pmw copyright

Copyright 1997-1999 Telstra Corporation Limited,
Australia Copyright 2000-2002 Really Good Software Pty Ltd, Australia

Permission is hereby granted, free of charge, to any person obtaining
a copy of this software and associated documentation files (the
"Software"), to deal in the Software without restriction, including
without limitation the rights to use, copy, modify, merge, publish,
distribute, sublicense, and/or sell copies of the Software, and to
permit persons to whom the Software is furnished to do so, subject to
the following conditions:

The above copyright notice and this permission notice shall be
included in all copies or substantial portions of the Software.

THE SOFTWARE IS PROVIDED "AS IS", WITHOUT WARRANTY OF ANY KIND,
EXPRESS OR IMPLIED, INCLUDING BUT NOT LIMITED TO THE WARRANTIES OF
MERCHANTABILITY, FITNESS FOR A PARTICULAR PURPOSE AND
NONINFRINGEMENT. IN NO EVENT SHALL THE AUTHORS OR COPYRIGHT HOLDERS BE
LIABLE FOR ANY CLAIM, DAMAGES OR OTHER LIABILITY, WHETHER IN AN ACTION
OF CONTRACT, TORT OR OTHERWISE, ARISING FROM, OUT OF OR IN CONNECTION
WITH THE SOFTWARE OR THE USE OR OTHER DEALINGS IN THE SOFTWARE.

"""
\end{verbatim}
\end{itemize}


\begin{abstract}

\noindent
Big Python is a special version of Python for users who require larger 
keys on their keyboards.  It accommodates their special needs by ...

\end{abstract}

\tableofcontents


\chapter{...}

My chapter.


\appendix
\chapter{...}

My appendix.

The \code{\e appendix} markup need not be repeated for additional
appendices.


%
%  The ugly "%begin{latexonly}" pseudo-environments are really just to
%  keep LaTeX2HTML quiet during the \renewcommand{} macros; they're
%  not really valuable.
%
%  If you don't want the Module Index, you can remove all of this up
%  until the second \input line.
%
%begin{latexonly}
\renewcommand{\indexname}{Module Index}
%end{latexonly}
\input{mod\jobname.ind}		% Module Index

%begin{latexonly}
\renewcommand{\indexname}{Index}
%end{latexonly}
\input{\jobname.ind}			% Index

\end{document}
