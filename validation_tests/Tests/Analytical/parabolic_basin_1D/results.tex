\section{1D Parabolic Basin}
This simulates flow oscillations in a 1D parabolic basin. The analytical solution is described in (REF), and is periodic. At any instant in time, the free surface elevation is planar, and the velocity is constant (in wet regions). The scenario includes regular wetting and drying, as the flow oscillates back and forth in the basin. As well as testing the ability of the code to do wetting and drying, it will highlight any numerical energy loss or gain, manifest as an increase or decrease in the magnitude of the flow oscillations over long time periods, compared with the analytical solution. 

\subsection{Results}
Figures~\ref{stagecent} and~\ref{stagewd} show the time-evolution of the water elevation at two points in the parabola. There should be a good agreement with the analytical solution, although as time goes on some deviations may appear.  

\begin{figure}[h]
\begin{center}
\includegraphics[width=0.9\textwidth]{Stage_centre_v2.png}
\caption{Stage over time in the centre of the parabola}
\label{stagecent}
\end{center}
\end{figure}

\begin{figure}[h]
\begin{center}
\includegraphics[width=0.9\textwidth]{Stage_centre_v3.png}
\caption{Stage over time at a point on the parabola where wetting and drying occurs}
\label{stagewd}
\end{center}
\end{figure}

Figures~\ref{velsnearpeak} and~\ref{velsnearzero} show the velocity profile at several instants in time, near to the absolute maximum and minimum respectively. There should be good agreement with the analytical solution for Figure~\ref{velsnearpeak}. For Figure~\ref{velsnearzero}, we expect velocities to be small, but accept more obvious numerical errors so long as this is the case. In previous versions of ANUGA, these plots would sometimes show strong velocity spikes near the wet dry boundaries. 

\begin{figure}[h]
\begin{center}
\includegraphics[width=0.9\textwidth]{Vel_3_5T_v2.png}
\caption{Velocity at an instant in time}
\label{velsnearpeak}
\end{center}
\end{figure}

\begin{figure}[h]
\begin{center}
\includegraphics[width=0.9\textwidth]{Vel_3T_v2.png}
\caption{Velocity at an instant in time}
\label{velsnearzero}
\end{center}
\end{figure}


\endinput
