\section{Deep Water Wave Propagtion}
This simulates the free propagation of a sinusoidal wave in deep water. Analytically, the wave should travel through the domain without deformation. However, in early versions of ANUGA, some config settings would lead to the wave dampening out too strongly. This sort of behaviour can have practical implications e.g. for tsunami propagation problems.  

This example can also illustrate difficulties with boundary conditions, as it is difficult to accurately treat the boundary where the wave exits the domain (of course, for this problem, we could do that by exploiting the analytical solution - but this is not possible for general wave propagation problems).

\subsection{Results}
Figure~\ref{fig:stagewave} shows the time-evolution of the water elevation at three points in the domain. These time series should show the wave propagating without deformation or attenuation (i.e. the wave has the same shape, amplitude, period, mean water level etc at each point).  
\begin{figure}[h]
\begin{center}
\includegraphics[width=0.9\textwidth]{wave_atten.png}
\caption{Stage over time at 3 points in space}
\label{fig:stagewave}
\end{center}
\end{figure}


The corresponding momentums of Figure~\ref{fig:stagewave} are shown in Figures~\ref{fig:xmom} and~\ref{fig:ymom}.
\begin{figure}[h]
\begin{center}
\includegraphics[width=0.9\textwidth]{xmom.png}
\caption{Xmomentum over time at 3 points in space}
\label{fig:xmom}
\end{center}
\end{figure}
\begin{figure}[h]
\begin{center}
\includegraphics[width=0.9\textwidth]{ymom.png}
\caption{Ymomentum over time at 3 points in space}
\label{fig:ymom}
\end{center}
\end{figure}



\endinput
