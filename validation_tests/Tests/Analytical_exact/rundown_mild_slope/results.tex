%Note for this test case:
%Simple water flow example using ANUGA: Water flowing down a channel.
%It was called "steep_slope" in an old validation test.


\section{Shallow flow down a mild slope}
This case simulates very shallow flow running down a mild slope topography. It represents an idealisation of the rainfall-runoff problem, which will often involve very shallow flows down such a topography. This case has an analytical solution (the steady-uniform solution, bed slope = friction slope).   

\subsection{Results}
Figures~\ref{fig:depthdownchan} shows the steady state depth in the downstream direction. There should be a good agreement with the analytical solution, at least away from the boundaries.  

\begin{figure}[h]
\begin{center}
\includegraphics[width=0.8\textwidth]{final_depth.png}
\caption{Depth in the downstream direction}
\label{fig:depthdownchan}
\end{center}
\end{figure}

Figures~\ref{fig:xvelscrosschan} and~\ref{fig:yvelscroschan} show the steady state $x$- and $y$-velocities, along a slice in the cross slope direction (near $x=50$). The $x$-velocities should agree well with the analytical solution, and the $y$-velocities should be zero.  

\begin{figure}[h]
\begin{center}
\includegraphics[width=0.8\textwidth]{x_velocity.png}
\caption{$x$-velocity along the cross-section $x=50$ (i.e. a cross-section with constant bed elevation)}
\label{fig:xvelscrosschan}
\end{center}
\end{figure}

\begin{figure}[h]
\begin{center}
\includegraphics[width=0.8\textwidth]{y_velocity.png}
\caption{$y$-velocity along the cross-section $x=50$ (i.e. a cross-section with constant bed elevation)}
\label{fig:yvelscroschan}
\end{center}
\end{figure}


\endinput
