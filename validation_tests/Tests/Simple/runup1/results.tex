\section{Simple Wave Runup}
This scenario simulates a wave flowing up a planar beach. Following the initial wave runup, eventually the water elevation should become constant, and the velocities should approach zero. 

\subsection{Results}
Figure~\ref{runupfig1} shows the velocity profile during the wave runup (in the cross-shore direction). The velocities should be free from major spikes.

\begin{figure}[h]
\begin{center}
\includegraphics[width=0.9\textwidth]{vel1d_1s_v2.png}
\caption{Velocity during the wave runup}
\label{runupfig1}
\end{center}
\end{figure}

Figure~\ref{runupfig2} shows the velocity profile at time 30s (in the cross-shore direction). It should be nearly zero (e.g. $<<$ 1 mm$/s$). This case has been used to illustrate wet-dry artefacts in some versions of \anuga.

\begin{figure}[h]
\begin{center}
\includegraphics[width=0.9\textwidth]{vel1d_30s_v2.png}
\caption{Velocity at time 30s after the wave runup. It should be nearly zero.}
\label{runupfig2}
\end{center}
\end{figure}

Figure~\ref{runupfig3} shows the water surface profile at time 30s (in the cross-shore direction). It should be nearly constant (= -0.1m) in the wet portions of the domain.
\begin{figure}[h]
\begin{center}
\includegraphics[width=0.9\textwidth]{elev_30s_v2.png}
\caption{Water elevation at time 30s after the wave runup. It should be nearly constant in wet parts of the domain.}
\label{runupfig3}
\end{center}
\end{figure}

\endinput
