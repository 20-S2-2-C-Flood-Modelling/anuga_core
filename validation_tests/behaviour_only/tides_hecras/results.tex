\section{Tides in HECRAS and \anuga{}}
This test compares idealised tidal channel flow in HECRAS and ANUGA. The tidal
amplitude (0.4m) is a significant fraction of the channel depth ($\simeq 1$m),
so nonlinear tidal deformation is significant.  

A 20m wide, 1000m long straight channel with rectangular cross-section flows through a
floodplain (10m wide on either side of the channel). Upstream of 220m the bed
elevation is constant (1m below the floodplain), downstream of 100m it is 3m below
the floodplain, and it varies linearly in between. The floodplain has a constant
elevation (1m). Manning's n is 0.03. The upstream boundary is reflective, while
at the downstream boundary a stage timeseries $y(t)$ is imposed (where $t$ is
time in seconds):
$$y=0.6+0.4*\sin(2\pi t/(1800))$$

\subsection{Results}
Figure~\ref{result} show stage timeseries at various stations downstream in each model.
For visual clarity the gauges are offset vertically.  The ANUGA and HECRAS
results should be visually indistinguishable.  There is a clear deformation of
the tide as it travels upstream, with the incoming tide having a shorter
duration than the outgoing tide at upstream stations, which is typical for
tides in shallow channels when overbank effects are not dominant. In this
example there is minor overbank inundation in parts of both models, but it is
just a few cm deep.

\begin{figure}
\begin{center}
\includegraphics[width=1.0\textwidth]{CENTRAL_CHANNEL.png}
\end{center}
\caption{Tidal levels in HECRAS and \anuga{}}
\label{result}
\end{figure}

Note that this ANUGA model uses a coarse mesh with discontinuous elevation to
resolve the banks, so this example should not run with non-discontinuous
elevation flow algorithms.

